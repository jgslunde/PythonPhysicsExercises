\documentclass[10pt,a4paper]{article}
 
\usepackage[utf8x]{inputenc}
\usepackage[norsk]{babel}
\usepackage[T1]{fontenc,url}
\usepackage[hang,small,bf]{caption}
\usepackage{relsize}
\usepackage{setspace}
\usepackage{parskip}
\usepackage{lmodern}
\usepackage{microtype}
\usepackage{verbatim}
\usepackage{amsmath, amssymb, amsthm}
\usepackage{mathtools}
\usepackage{tikz}
\usepackage{physics}
\usepackage{algorithm}
\usepackage{algpseudocode}
\usepackage{listings}
\usepackage{enumerate}
\usepackage{graphicx}
\usepackage{float}
\usepackage{hyperref}
\usepackage{varioref}
\usepackage{siunitx}
\usepackage{todonotes}
\usepackage{color}
\usepackage[margin=3cm]{geometry}
\labelformat{equation}{ligning~(#1)}
 
\renewcommand{\exp}{\mathrm{e}^}
\newcommand{\halflife}{t_{\frac{1}{2}}}
\newcommand{\half}{\frac{1}{2}}
\newcommand{\planck}{$h = \SI{6.626e-34}{J.s}$}
 
\definecolor{light_green}{rgb}{0, 0.6, 0}
\definecolor{light_grey}{rgb}{0.5, 0.5, 0.5}
\definecolor{magenta}{rgb}{0.7, 0, 0.5}
 
 
\lstdefinestyle{py}{
    language = python,
    frame = single,
    showstringspaces = false,
    basicstyle = \small\ttfamily,
    breaklines = true,
    commentstyle = \color{light_grey},
    keywordstyle = \color{magenta},
    stringstyle = \color{light_green},
}
 
 
 
\begin{document}
	\section*{Oppgave 1.1 - Massetetthet}
	\addcontentsline{toc}{section}{Oppgave 1.1 - Massetetthet - \texttt{massdensity.py}}
	Ulike materiale har ulike massetettheter. Massetettheten er definert masse delt på volum, oftest i $\mathrm{kg}/\mathrm{m^3}$.   
	\begin{center}
	\begin{tabular}{l | l  l  l  l l}
			Materialer& Polystyren (lav tetthet) & Kork & Rhenium & Platinium \\ \hline 
			Massetetthet (i \si{\kg.\per\cubic\meter}) &20 & 220  & 21020 & 21450
	\end{tabular} 
	\captionof{table}{Massetettheten til ulike materialer}
	\end{center}
	En kube veier \SI{858}{g} og har volum \SI{40}{cm^3}. Skriv et program som finner massetettheten til kuben og bruk resultatet til å finne hvilket materiale kuben består av. 
	Kuben består av et av materialene i tabellen over. Det kan hende at svaret ditt vil ikke være den eksakt en av de gitte massetetthetene siden massetetthet er i seg selv et gjennomsnittlig mål. Du kan velge materialet som har massetett som er nærmest ditt resultat. 
	
	Filnavn: \texttt{massdensity.py} 
	\section*{Oppgave 1.2 - Beregne solmassen}
	\addcontentsline{toc}{section}{Oppgave 1.2 - Beregne solmassen - \texttt{solarmass.py}}
	Det er mulig å regne ut solens masse ved å bruke at 
	\begin{equation}\label{eq:solarmass}
	M_{sol} = \frac{4\pi^2 \cdot\qty(\SI{1}{AU})^3}{G \cdot \qty(\SI{1}{yr})^2}
	\end{equation}
	I denne oppgaven kommer vi til å bruke omtrentlige verdier for AU og G. \\
	Enheten AU er en astronomisk lengdeenhet gitt ved avstanden mellom solen og jorden. Den har en verdi
	\[
	\SI{1}{AU} = \SI{1.58e-5 }{\text{lysår}}
	\]
	der
	$
	\SI{1}{lysår} = \SI{9.5e12}{\km} 
	$.
	
	Konstanten G kalles gravitasjonskonstanten og har verdi 
	\[
	G = \SI{6.674e-11}{\cubic\meter\per\kg\per\second} 
	\]
 
	Lag et program som regner ut solmassen ved å bruke \vref{eq:solarmass} og som skriver ut resultatet i kilogram med en pent formatert \texttt{print}.  Resultatet ditt skal bli omtrent $M_{sol} \approx 2.01 \cdot 10^{30}\,$kg. 
	
	
	Filnavn: \texttt{solarmass.py}
	
 
 
\section*{Oppgave 1.3 - Halveringstid}
\addcontentsline{toc}{section}{Oppgave 1.3 - Halveringstid - \texttt{half\_life.py}}
Et radioaktivt materie er et ustabilt stoff som over tid vil reduseres til andre materier mens det avgir radioaktiv stråling. Av en opprinnelig masse $N_0$ av et radioaktivt materie, vil den gjenværende massen etter en tid $t$ være gitt ved
\begin{align}
N(t) = N_0\exp{-t/\tau}
\end{align}
$\tau$ er den såkalte 'tidskonstanten' til det radioaktive materiet, og representerer den gjennomsnittlige tiden et enkelt atom bruker før det nedbrytes. Den kan variere fra under $\SI{e-20}{s}$ for ekstremt ustabile isotoper, til over $\SI{e30}{s}$ for stoffer som for alle praktiske formål kan regnes som stabile.
 
 
\subsection*{a)}
Karbon-11 er et ustabilt karbonisotop, og har en tidskonstant $\tau = 1760\; \mathrm{s}$.
 
Lag et program som beregner hvor mye som gjenstår av karbonet etter 10 minutter. Anta at vi starter med en mengde $N_0 = \SI{4.5}{kg}$ av karbon-11.
 
\textbf{Hint:} For å teste at programmet ditt fungerer som det skal, kan du sette $t$ lik $0$, og et meget stort tall, og sjekke at resultatene virker i overensstemmelse med slik du forventer et radioaktiv materiale å oppføre seg.
 
 
\subsection*{b)}
Selv om tidskonstanten gir oss en veldig pen formel, snakker vi oftere om 'halveringstiden' $\halflife$ til materiet, som er tiden det tar for halvparten av materiet å nedbrytes. Forholdet mellom tidskonstanten og halveringstiden er gitt ved
\begin{align*}
\tau = \frac{\halflife}{\ln 2}
\end{align*}
 
Halveringstiden til karbon-11 er $\halflife = 1220\;\mathrm{s}$. Skriv om programmet ditt slik at det først regner ut tidskonstanten fra halveringstiden, og deretter beregner mengden av stoffet, som i forrige oppgave. \\
Kontroller at du får samme svar som i forrige oppgave dersom du bruker samme masse og tid.
 
Filnavn: \texttt{half\_life.py}
 
 
 
	\section*{Oppgave 1.4 - Farten til et atom}
	\addcontentsline{toc}{section}{Oppgave 1.4 - Farten til et atom - \texttt{velocity\_of\_atom.py}}
	Atomene i et materiale er strukturert slik at de danner et repeterende gitter mønster av deres plasseringer. 
	Vi skal se på et atom som beveger seg langs overflaten til materialet. På grunn av gitterstrukturen, kan vi lage en modell der farten til atomet er periodisk:
	\begin{equation*}
		v(x) = \sqrt{v_0^2 + \frac{2F_0}{m}\qty(\cos\qty(\frac{x}{n}) -1 )}
	\end{equation*}
	der $m$ er atomets masse, $x$ er posisjonen til atomet, $v_0$ er atomets startfart og $n$ er en skalert avstand mellom atomene i materialet. Vi setter kraften $F_0 = 1$.\\
	Finn farten til atomet når $x = 1$, $v_0 = 2$, $n = 4$ og $m = 3$.
	
	Filnavn: \texttt{velocity\_of\_atom.py}
 
\section*{Oppgave 1.5 - Finn feil hos Einstein}
	\addcontentsline{toc}{section}{Oppgave 1.5 - Finn feil hos Einstein -  \texttt{Einsteins\_errors.py}}
Spesiell relativitetsteori er et område i fysikken som omhandler veldig store hastigheter. I spesiell relativitetsteori er bevegelsesmengden $p$ til et objekt med masse $m$ (i kg) og hastighet $v$ (i m/s) gitt ved
\begin{align*}
p = m\cdot v\cdot \gamma, \ \ \ \ \gamma = \frac{1}{\sqrt{1-\frac{v^2}{c^2}}}
\end{align*}
der $c\approx \SI{300000000}{m/s}$ er lyshastigheten. Programmet under forsøker å regne ut bevegelsesmengden til et objekt med en hastighet på $1/3$ av lyshastigheten, med masse $m=0.14\;\mathrm{kg}$. Programmet er fullt av feil. Kopier programmet og kjør det. Rett opp feilene, og få programmet til å printe en korrekt $p$.
\\
\lstinputlisting[language=Python, frame=single, numbers=left, showstringspaces=false]{find_Einsteins_errors.py}
 
Filnavn: \texttt{Einsteins\_errors.py}
 
\section*{Oppgave 1.6 - Rydbergs konstant}
\addcontentsline{toc}{section}{Oppgave 1.6 - Rydbergs konstant- \texttt{constant\_Rydberg.py}}

Rydbergkonstanten $R_{\infty}$ for tunge atomer brukes i fysikken for å kunne beregne bølgelengden til spektrallinjer\footnote{Årsaken til at vi bruker $\infty$ i symbolet til konstanten er fordi vi antar at massen til atomets kjerne er uendelig stor sammenlignet med massen til elektronet.  }. 
 
Konstanten er funnet til å være
\[
R_{\infty} = \frac{m_ee^4}{8 \varepsilon_0^2h^3c }
\]
der 
\begin{itemize}
	\item $m_e = \SI{9.109e-31}{\meter}$ er massen til elektronet 
	\item $e = \SI{1.602e-19}{C}$ er ladningen til et proton (også kalt for \textit{elementærladningen})
	\item $\varepsilon_0 = \SI{8.854e-12}{C.V^{-1}.\meter^{-1}}$ er den elektriske konstanten
	\item $h = \SI{6.626e-34}{J.s}$ er Plancks konstant
	\item $c = \SI{3e8}{\meter.\per\second}$ er lysets fart
\end{itemize}
Dette er fysiske konstanter som brukes en god del i fysikk, og som du høyst sannsynlig kommer til å møte på flere ganger i andre fysikkemner!
 
Lag et program som setter de fysiske konstantene som variabler og bruker variablene til å regne ut verdien til Rydbergs konstant. Verdien til Rydbergskonstant skal da skrives til skjerm ved hjelp av \texttt{print}. 
 
Se om programmet ditt gir at
\[
R_{\infty} = 10961656.2162\quad (\text{i }\mathrm{m^{-1}})
\]
Merk at dette er kun en omtrentlig verdi. Dette er fordi vi har rundet av verdiene de fysiske konstantene til tre desimaler. 
 
Filnavn: \texttt{constant\_Rydberg.py}

\end{document}
 
 

