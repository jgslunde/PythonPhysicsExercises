\documentclass[10pt,a4paper]{article}

\usepackage[utf8x]{inputenc}
\usepackage[norsk]{babel}
\usepackage[T1]{fontenc,url}
\usepackage[hang,small,bf]{caption}
\usepackage{relsize}
\usepackage{setspace}
\usepackage{parskip}
\usepackage{lmodern}
\usepackage{microtype}
\usepackage{verbatim}
\usepackage{amsmath, amssymb, amsthm}
\usepackage{mathtools}
\usepackage{tikz}
\usepackage{physics}
\usepackage{algorithm}
\usepackage{algpseudocode}
\usepackage{listings}
\usepackage{enumerate}
\usepackage{graphicx}
\usepackage{float}
\usepackage{hyperref}
\usepackage{varioref}
\usepackage{todonotes}
\usepackage{color}
\usepackage{siunitx}
\usepackage[margin=1.5cm]{geometry}
\labelformat{equation}{ligning~(#1)}

\renewcommand{\exp}{\mathrm{e}^}
\newcommand{\halflife}{t_{\frac{1}{2}}}

\definecolor{light_green}{rgb}{0, 0.6, 0}
\definecolor{light_grey}{rgb}{0.5, 0.5, 0.5}
\definecolor{magenta}{rgb}{0.7, 0, 0.5}


\lstdefinestyle{py}{
    language = python,
    frame = single,
    showstringspaces = false,
    basicstyle = \small\ttfamily,
    breaklines = true,
    commentstyle = \color{light_grey},
    keywordstyle = \color{magenta},
    stringstyle = \color{light_green},
}

\begin{document}


\section*{Oppgave 1.1}
Her er det viktig at de passer på å konvertere til riktige enheter, og unngår heltallsdivisjon. Materialtettheten til kuben er 21450.00 \si{\kg.\per\cubic\meter}, hvilket som betyr at kuben er laget av Platinium. 
\lstinputlisting[style=py]{massdensity.py}



\section*{Oppgave 1.2}
Igjen er det viktig de passer på å få riktige enheter. En riktig så skjult 'felle', er at et år må konverteres til sekunder. 
\lstinputlisting[style=py]{solarmass.py}

\section*{Oppgave 1.3}
Pass på heltallsdivisjon i \texttt{-t/tau}, og at 10 minutter blir konvertert til sekunder. Kanskje noen også kunne funnet på å bruke tallet \texttt{e}, istedenfor math modulens \texttt{exp}, noe som ikke er feil men heller ikke skal gjøres.
\lstinputlisting[style=py]{half_life.py}

\begin{verbatim}
>>> Remaining weight of carbon-11 is 3.2 kg
\end{verbatim}

\newpage
\section*{Oppgave 1.4}
Riktig bruk av \texttt{math}-modulen og unngå heltallsdivisjon.
\lstinputlisting[style=py]{velocity_of_atom.py}
Resultat:
\begin{verbatim}
>>> The velocity of the atom at position x = 1 is v = 1.99481
\end{verbatim}

\section*{Oppgave 1.5}
Linje 1 og 6: Math-modulen sin kvadratrot heter "sqrt". Det er eventuelt mulig å navngi importerte funksjoner med følgende syntax: \texttt{from math import sqrt as squareroot}

Linje 2 og 3: Selv om det til daglig ofte er vanlig å separere hvert tredje tall med et mellomrom, godtar ikke python syntaxen dette. Det er også mest naturlig å skrive såpas store tall med vitenskapelig notasjon.

Linje 4: For desimaltall i Python må vi bruke et punktum. Komma brukes for å separere elementer.

Linje 6: I Python skrives eksponenter som '**'.

Linje 6, eller 2 og 3: Det forekommer en heltallsdivisjon mellom $v^2$ og $c^2$. Vi kan enten initialisere variablene som floats i linje 2 og 3, eller konvertere dem til floats før vi gjør divisjonen i linje 6. Det mest naturlig er det førstenevnte, som er gjort i koden under.

Kravet for å beste oppgaven er å ha fått den til å fungere. Koden skal se ut som noe slikt:
\lstinputlisting[style=py]{find_Einsteins_errors_solution.py}

\section*{Oppgave 1.6}
Hensikten med dette oppgaven er at de skal få se nytten i å lagre verdier som variabler. Det er derfor viktig at passende variabler blir tilordnet passende verdier. 
\lstinputlisting[style=py]{constant_Rydberg.py}
Resultat:
\begin{verbatim}
>>> The Rydberg constant is approximately: 1.09617e+07 m^{-1}
\end{verbatim}
\end{document}


