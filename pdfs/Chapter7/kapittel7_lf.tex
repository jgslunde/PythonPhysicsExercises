\documentclass[10pt,a4paper]{article}

\usepackage[utf8x]{inputenc}
\usepackage[norsk]{babel}
\usepackage[T1]{fontenc,url}
\usepackage[hang,small,bf]{caption}
\usepackage{relsize}
\usepackage{setspace}
\usepackage{parskip}
\usepackage{lmodern}
\usepackage{microtype}
\usepackage{verbatim}
\usepackage{amsmath, amssymb, amsthm}
\usepackage{mathtools}
\usepackage{tikz}
\usepackage{physics}
\usepackage{algorithm}
\usepackage{algpseudocode}
\usepackage{listings}
\usepackage{enumerate}
\usepackage{graphicx}
\usepackage{float}
\usepackage{hyperref}
\usepackage{varioref}
\usepackage{siunitx}
\usepackage{todonotes}
\usepackage{color}
\usepackage[margin=1cm]{geometry}
\labelformat{equation}{equation~(#1)}

\renewcommand{\exp}{\mathrm{e}^}
\newcommand{\halflife}{t_{\frac{1}{2}}}
\newcommand{\half}{\frac{1}{2}}
\newcommand{\planck}{$h = \SI{6.626e-34}{J.s}$}

\definecolor{light_green}{rgb}{0, 0.6, 0}
\definecolor{light_grey}{rgb}{0.5, 0.5, 0.5}
\definecolor{magenta}{rgb}{0.7, 0, 0.5}


\lstdefinestyle{py}{
    language = python,
    frame = single,
    showstringspaces = false,
    basicstyle = \small\ttfamily,
    breaklines = true,
    commentstyle = \color{light_grey},
    keywordstyle = \color{magenta},
    stringstyle = \color{light_green},
}


\begin{document}

\section*{Oppgave 7.1}
Husk at 'self' parameteren må være med i alle metoder, selv om de ikke tar noen argumenter.

Jeg kan forestille meg at det er mange måter å mislykkes i å tildele en attribute til en klasse, men enten funker det eller ikke. Det er selvsagt ikke lov å endre klasse-koden til å ha en ny attribute...
\lstinputlisting[style=py]{Planet.py}


\newpage
\section*{Oppgave 7.2}
\subsection*{a)}
Her er det viktig at de får definert funksjonen som beskrevet i oppgaven ordentlig. Det er viktig de sender inn en partikkelinstanse som parameter for å regne ut Coulumbs lov for at studentene skal lære seg å hente ut attributter. For å finne avstanden mellom to partikler har \texttt{np.linalg.norm} blitt foreslått ,men andre metoder som finner den euklidske normen mellom posisjonene er også helt greit.
\subsection*{b)}
En helt standard testfunksjon der partikkelinstanser opprettes og testes. 
\lstinputlisting[style=py]{Particle_Coulomb.py}

\newpage
\section*{Oppgave 7.3}
Dekomponeringen gir at\\
$x(t) = x_0 + v_{x0}t$\\
$y(t) = y_0 + v_{y0}t + 0.5at^2$\\
$v_x(t) = v_{x0}$\\
$v_y(t) = v_{y0} + at$

Testfunksjonen skal bare velge en t-verdi, sammenligne returen fra de innebyggede metodene med beregnede verdier.

Energi testfunksjonen velger to t-verdier, og sammenligner resultatet av summen av de to oppgitte energifunksjonene.
\lstinputlisting[style=py]{UFO.py}


\newpage
\section*{Oppgave 7.4}
Det nye er å implementere og bruke en \texttt{\_\_str\_\_}-funksjon. Viktig at de passer på at helltallsdivisjon ikke oppstår når de bruker instanser av klassen.
\lstinputlisting[style=py]{Runner.py}
Resultat etter kjøring av denne koden:
\begin{verbatim}
>>> Sprinter with 
			      mass: 80 kg 
			      initial velocity: 5 m/s
			      angle: 30 degrees
>>> Finish runtime for distance 100 m: 4.017 s
\end{verbatim}



\newpage
\section*{Oppgave 7.5}
Her er hovedfokuset i å bruke instanser og utføre beregner på dem utenfor klassen. 
\lstinputlisting[style=py]{Center_of_mass.py}
Resultat:
\begin{verbatim}
>>> Center of mass in this system is:
		[ 3.  6.]
\end{verbatim}
\end{document}


