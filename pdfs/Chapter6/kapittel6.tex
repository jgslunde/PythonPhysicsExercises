\documentclass[10pt,a4paper]{article}
 
\usepackage[utf8x]{inputenc}
\usepackage[norsk]{babel}
\usepackage[T1]{fontenc,url}
\usepackage[hang,small,bf]{caption}
\usepackage{relsize}
\usepackage{setspace}
\usepackage{parskip}
\usepackage{lmodern}
\usepackage{microtype}
\usepackage{verbatim}
\usepackage{amsmath, amssymb, amsthm}
\usepackage{mathtools}
\usepackage{tikz}
\usepackage{physics}
\usepackage{algorithm}
\usepackage{algpseudocode}
\usepackage{listings}
\usepackage{enumerate}
\usepackage{graphicx}
\usepackage{float}
\usepackage{hyperref}
\usepackage{varioref}
\usepackage{siunitx}
\usepackage{todonotes}
\usepackage{color}
\usepackage[margin=3cm]{geometry}
\labelformat{equation}{ligning~(#1)}
 
\renewcommand{\exp}{\mathrm{e}^}
\newcommand{\halflife}{t_{\frac{1}{2}}}
\newcommand{\half}{\frac{1}{2}}
\newcommand{\planck}{$h = \SI{6.626e-34}{J.s}$}
 
\definecolor{light_green}{rgb}{0, 0.6, 0}
\definecolor{light_grey}{rgb}{0.5, 0.5, 0.5}
\definecolor{magenta}{rgb}{0.7, 0, 0.5}
 
 
\lstdefinestyle{py}{
    language = python,
    frame = single,
    showstringspaces = false,
    basicstyle = \small\ttfamily,
    breaklines = true,
    commentstyle = \color{light_grey},
    keywordstyle = \color{magenta},
    stringstyle = \color{light_green},
}
 
 
\begin{document}
\section*{Oppgave 6.1 - Solsystem}
\addcontentsline{toc}{section}{Oppgave 6.1 - Solsystem - \texttt{solar\_system\_dict.py}}
Følgende fil inneholder informasjon om gjennomsnittlig avstand fra solen, masse, og radius ved ekvator, til en rekke himmellegemer i vårt solsystem.
 
\lstinputlisting[frame=single]{solar_system_data.dat}
 
\subsection*{a)}
Skriv et program som leser filen \texttt{solar\_system\_data.dat}, og skriver verdiene inn i tre dictionaries \texttt{distance}, \texttt{mass} og \texttt{radius}. 
 
\subsection*{b)}
Lag et nytt dictionary, \texttt{density}, som inneholder massetettheten til hver av objektene i $\mathrm{kg/m^3}$. Du kan anta at alle planetene er perfekte sfærer. Volumet til en sfære er $V = \frac{4}{3}\pi r^3$.
 
\subsection*{c) (Valgfri ekstraoppgave)}
Kombiner informasjonen i et nøstet dictionary, \texttt{solar\_system} slik at f.eks. \texttt{solar\_system[Jupiter][radius]} gir radiusen til Jupiter.
 
Test at det nye dictionary'et du har laget fungerer som det skal ved å regne ut hvor mange jordkloder det tar å rekke fra solen og til Pluto.
 
Filnavn: \texttt{solar\_system\_dict.py}
 
 
 
 
 
 
 
\section*{Oppgave 6.2 - Lese og bruke fysiske konstanter}
\addcontentsline{toc}{section}{Oppgave 6.2 - Lese og bruke fysiske konstanter - \texttt{constants\_hydrogen.py}}
Det er ofte vi har behov for en god del fysiske konstanter for å modellere ulike fysiske systemer. 
\subsection*{a)}
Lag et program som leser \texttt{physics\_constant.dat}. Filen har et slikt format slik at navnene (av vilkårlig lengde) står foran et kolon '\texttt{:}'. Deretter, står konstantens symbol, så verdien og til slutt konstantens enhet.  Programmet skal lagre konstantene i et dictionary. Konstantenes symbol skal brukes som nøkkel for å hente ut konstantens tilhørende verdi. 
\subsection*{b)}
Bohr utledet en formel på hvordan en kan modellere energinivåene $E_n$ til et hydrogenatom. Energinivåene er utledet til å være
\[
E_n =  -\frac{k_e^2m_ee^4}{2\hbar^2} \frac{1}{n^2}
\]
Bruk dictionary-et fra a) for å hente ut de nødvendige konstantene som brukes i modellen, og se om programmet ditt får at 
\begin{align*}
 \frac{k_e^2m_ee^4}{2\hbar^2} &\simeq \SI{2.18e-18}{\joule}\\
 &\simeq \SI{13.6}{\electronvolt} 
\end{align*}
der $k_e$ er Coulombs konstant, $m_e$ elektronets masse, $e$ er elementærladningen og $\hbar$ er Plancks reduserte konstant (du finner den i filen som \texttt{hbar}). 
 
Filnavn: \texttt{constants\_hydrogen.py}
 
\section*{Oppgave 6.3 - Dynamisk friksjon}
\addcontentsline{toc}{section}{Oppgave 6.3 - Dynamisk friksjon - \texttt{dynamic\_friction\_pair.py}}
Når et legeme beveger seg på et underlag, virker det en dynamisk friksjonskraft på legemet. Friksjonskraften er dynamisk fordi legemet er i bevegelse. 
 
Den dynamiske friksjonskraften $\mu_D$ er
\[
\mu_D = N\mu
\]
der $\mu$ er friksjonstallet mellom materialene legemet og underlaget består av ($\mu$ vil variere ettersom hvilken materiale-par vi ser på) og $N$ er normalkraften som virker på legemet\footnote{Normalkraften er den kraften som virker på legemet fra underlaget der legemet og underlaget er i kontakt med hverandre.}.
 
Vi antar legemet beveger seg i horisontal retning. Dette gir at 
\[
N = mg
\]
der $m$ er legemets masse og $g = 9.81\,\si{\meter.\per\square\second}$. 
 
Vi skal se på hva den dynamiske friksjonen blir ettersom hvilket materiale legemet og underlaget er laget av. 
 
\subsection*{a)}
Lag et program som leser inn filen \texttt{friction\_coefficients\_data.dat} (som du kan finne her: lenke-til-fil) som er en tabell som består av friksjonstall for noen materiale-par. 
 
Den første linjen forteller hvilket material-par vi ser på, og den andre linjen er tilhørende friksjonstall til material-paret. 
 
Programmet skal hente ut og lage separerte lister over material-parene og friksjonstallene. Programmet ditt trenger ikke å ta hensyn til bindestreken mellom materialene (enn så lenge). 
 
\subsection*{b)}
Nå skal programmet bruke listene av material-parene og friksjonstallene.
 
For hvert materialpar la programmet hente ut materialet til legemet (materialet før bindestreken),materialet til underlaget (etter bindestreken) og tilhørende friksjonstall $\mu$ til paret. La programmet beregne den dynamiske friksjonskraften. Kraften skal så skrives til skjerm der det kommer tydelig fram hvilket materiale legemet og overflaten er laget av. 
 
Her lar vi legemet ha masse $m = 2.5\,\si{\kg}$.
 
Filnavn: \texttt{dynamic\_friction\_pair.py}
 
\section*{Oppgave 6.4 - På Jorden}
\addcontentsline{toc}{section}{Oppgave 6.4 - På Jorden - \texttt{different\_g.py}}
Vi har arbeidet mye med tyngdeakselerasjon  $g = 9.81\,\mathrm{m/s^2}$ uten å bekymre oss alt for mye om hvor vi befinner oss på Jorden. Faktisk, så er $g$ avhengig av hvor vi befinner oss (eller; hvor mange meter vi er over havet og ved hvilken breddegrad vi befinner oss)\footnote{Dette er på grunn av sentrifugalkraften - en kraft som vi oppfatter som en tilstedeværende kraft på grunn av Jordens rotasjon}!

I filen  \texttt{data\_different\_g.dat} kan du finne verdier for $g$ (avrundet) for et utvalg byer. 
\section*{a)}
Skriv et program som leser filen \texttt{data\_different\_g.dat} og lager et dictionary der verdien av $g$ er lagret for hver by. 

For eksempel, er tyngdeakselerasjonen i Stockholm lik $g = 9.818$. Ditt dictionary burde være laget slik at hvis du sender \texttt{Stockholm} som nøkkel til det, burde tyngeakselerasjonen $ g = 9.818$ bli returnert. 

Vi skal anta at en city har enten ett mellomrom i sitt navn (for eksempel San Francisco), eller ingen (for eksempel Paris).

Den siste linjen som inneholder kun '\texttt{-}'s markerer slutten på listen over byene. 

\textbf{Hint: } Bruk \texttt{split()} for hver linje for å få en liste av strings som opprinnelig var separert av mellomrom. Dersom et navn inneholder ett mellomrom, så vil listen bli 1 enhet lengre enn listen med navn uten mellomrom. En kan derfor sjekke lengden til listen for å se om bynavnet har et mellomrom eller ei.  

Slicing vil også være veldig nyttig for å hente ut verdier for $g$. 
\section*{b)}
Utvid programmet ditt fra a) slik at den leser filen \texttt{on\_Earth.dat}. 
Filen inneholder flere linjer, hvorav hver av linjene består av to byer seperert med \texttt{to}. 

For hver by i \texttt{on\_Earth.dat}, må programmet ditt hente ut navnet av hver by og bruke dictionary-et fra a) for å skrive ut tyngdeakselerasjonen til byene sepearert med \texttt{to} (i samme format som i \texttt{on\_Earth.dat}, bare med tilhørende verdier av $g$ for hver by). 

Dette viser hvordan utskriften skal se ut:
\begin{verbatim}
9.813 to 9.8
9.805 to 9.816
9.82 to 9.78
\end{verbatim}

Den siste linjen i filen markerer slutten. Du kan anta at ingen bynavn begynner på '\texttt{On}'. 

\textbf{Hint: } Igjen er \texttt{split()} din venn. Utnytt antakelsen at bynavnene består enten av ett mellomrom eller ingen, ingen by har bynavn som begynner på '\texttt{On}' og det er alltid en \texttt{to} som separerer byene. 

For den andre byen (etter '\texttt{to}'), kan en bruke  \texttt{' '.join(line[start:])} for å hente ut dets hele navn. Indeksen \texttt{start} er avhengig om hvorvidt den første byen har et navn med eller uten ett mellomrom. 

Filename: \texttt{different\_g.py}
 
\end{document}
 
 

