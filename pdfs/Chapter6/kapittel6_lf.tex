\documentclass[10pt,a4paper]{article}

\usepackage[utf8x]{inputenc}
\usepackage[norsk]{babel}
\usepackage[T1]{fontenc,url}
\usepackage[hang,small,bf]{caption}
\usepackage{relsize}
\usepackage{setspace}
\usepackage{parskip}
\usepackage{lmodern}
\usepackage{microtype}
\usepackage{verbatim}
\usepackage{amsmath, amssymb, amsthm}
\usepackage{mathtools}
\usepackage{tikz}
\usepackage{physics}
\usepackage{algorithm}
\usepackage{algpseudocode}
\usepackage{listings}
\usepackage{enumerate}
\usepackage{graphicx}
\usepackage{float}
\usepackage{hyperref}
\usepackage{varioref}
\usepackage{todonotes}
\usepackage{color}
\usepackage{siunitx}
\usepackage[margin=1.5cm]{geometry}
\labelformat{equation}{ligning~(#1)}

\renewcommand{\exp}{\mathrm{e}^}
\newcommand{\halflife}{t_{\frac{1}{2}}}

\definecolor{light_green}{rgb}{0, 0.6, 0}
\definecolor{light_grey}{rgb}{0.5, 0.5, 0.5}
\definecolor{magenta}{rgb}{0.7, 0, 0.5}


\lstdefinestyle{py}{
    language = python,
    frame = single,
    showstringspaces = false,
    basicstyle = \small\ttfamily,
    breaklines = true,
    commentstyle = \color{light_grey},
    keywordstyle = \color{magenta},
    stringstyle = \color{light_green},
}


\begin{document}

\section*{Oppgave 6.1}
Filen må vellykket være lest, og all informasjonen må være samlet i tre dictionaries med planet-navnene som keys. Om noen mot formodning skulle klare å regne massetetthet feil så er ikke det så viktig.

Til oppgave c) er det selvsagt foretrukket om det brukes printf formatering til å få all dataen i rette kolonner, men det er godkjent så lenge det er minst et space mellom dataen på hver linje (ettersom det da er lesbart av et program).

\lstinputlisting[style=py]{solar_system_dict.py}

\begin{verbatim}
>>> Number of earths from Sun to Pluto = 925054.88
\end{verbatim}
\newpage	
\section*{Oppgave 6.2}
\subsection*{a)}
Viktig at symbolene til konstantene hentes ut og brukes som nøkkel til tilhørende verdier. For å hente ut nødvendig informasjon, er ganske \texttt{split()} greiest å bruke. Siden navnene på konstantene er av vilkårlig lengde, må programmet hente ut verdier bakfra i listen etter \texttt{split()} har blitt kalt.
\subsection*{b)}
Poenget er nå at de skal bruke dictionary-et til å hente ut de nødvendige verdiene for å utføre beregningen beskrevet i oppgaven. Fint om de er påpasselige på hvilken enhet verdien de skriver ut er i.  
\lstinputlisting[style=py]{constants_hydrogen.py}
Resultat:
\begin{verbatim}
>>> In V: 2.17987e-18
    In eV: 13.6057
\end{verbatim}
\newpage
\section*{Oppgave 6.3}
Her er det viktig at de behersker \texttt{split()} for å få de nødvendige verdiene i lister. Det er også nyttig i oppgaven å dele stringsene etter et gitt argument til \texttt{split} istedenfor mellomrom.
\lstinputlisting[style=py]{dynamic_friction_pair.py}
Resultat:
\begin{verbatim}
Material of object | Material of surface | Dynamic friction
steel                 steel                14.715 
steel                 ice                   1.22625 
ice                   ice                   0.73575 
ice                  rubber                 0.4905 
wood                  wood                  7.3575 
nickel                glass                19.1295 
wood                  stone                 6.13125 
steel                 plexiglass           11.0363 
diamond               metal                 2.4525 
\end{verbatim}
\newpage
\section*{Oppgave 6.4\footnote{Referanse til Samael - On Earth}}
\subsection*{a)}
Viktig å hente ut riktig bynavn, og passe på at programmet tar hensyn til navn med eller uten mellomrom. 
\subsection*{b)}
Her er det viktig å hente ut riktig bynavn, og hente ut $g$ for hver tilhørende by med dictionaryet fra a). Utfordringen blir nok å vite hvordan testene skal foregå. 
\lstinputlisting[style=py]{different_g.py}

\end{document}


