\documentclass[10pt,a4paper]{article}

\usepackage[utf8x]{inputenc}
\usepackage[norsk]{babel}
\usepackage[T1]{fontenc,url}
\usepackage[hang,small,bf]{caption}
\usepackage{relsize}
\usepackage{setspace}
\usepackage{parskip}
\usepackage{lmodern}
\usepackage{microtype}
\usepackage{verbatim}
\usepackage{amsmath, amssymb, amsthm}
\usepackage{mathtools}
\usepackage{tikz}
\usepackage{physics}
\usepackage{algorithm}
\usepackage{algpseudocode}
\usepackage{listings}
\usepackage{enumerate}
\usepackage{graphicx}
\usepackage{float}
\usepackage{hyperref}
\usepackage{varioref}
\usepackage{todonotes}
\usepackage{color}
\usepackage{siunitx}
\usepackage[margin=1.5cm]{geometry}
\labelformat{equation}{ligning~(#1)}

\renewcommand{\exp}{\mathrm{e}^}
\newcommand{\halflife}{t_{\frac{1}{2}}}

\definecolor{light_green}{rgb}{0, 0.6, 0}
\definecolor{light_grey}{rgb}{0.5, 0.5, 0.5}
\definecolor{magenta}{rgb}{0.7, 0, 0.5}


\lstdefinestyle{py}{
    language = python,
    frame = single,
    showstringspaces = false,
    basicstyle = \small\ttfamily,
    breaklines = true,
    commentstyle = \color{light_grey},
    keywordstyle = \color{magenta},
    stringstyle = \color{light_green},
}



\begin{document}

\section*{Oppgave 3.1}
Alle tre variablene skal sendes inn til funksjonen, ingenting skal være globalt. Heltallsdivisjon kan skje i \texttt{-t/tau}.

Test funksjonen bør helst inneholde en \texttt{assert} statement, og som nevnt i oppgaven kan ikke toleransen være mindre enn \texttt{1e-4} fordi \SI{3.2}{kg} er en (ganske nøyaktig) tilnærming.
\lstinputlisting[style=py]{radioactive_function.py}



\section*{Oppgave 3.2}
Her skal altså to \texttt{n}'er, \texttt{L} og \texttt{m} tas inn som variabler, og det skal regnes ut differansen mellom to energinivåer. 
\lstinputlisting[style=py]{quantum_function.py}



\section*{Oppgave 3.3}
Fokuset i oppgaven er at de får til å iterere gjennom den innsendte listen av friksjonskoeffsienter, utføre beregninger på dem og lagre resultatene i en liste. 

Det er ikke så viktig at \texttt{zip} brukes, men fint om de får vite at den finnes og hint til hvordan den kan brukes i denne oppgaven. 
\lstinputlisting[style=py]{block_frictions.py}
\newpage
Resultat etter kjøring av løsningsforslaget:
\begin{verbatim}
>>> Length: 2.05518 with frictioncoefficient: 0.62 
>>> Length: 4.24737 with frictioncoefficient: 0.3 
>>> Length: 2.83158 with frictioncoefficient: 0.45 
>>> Length: 6.37105 with frictioncoefficient: 0.2
\end{verbatim}



\section*{Oppgave 3.4}
Her er det viktig at de får til å lage en if-test i en funksjon (til å beregne poengene) uten å bruke globale variable. 
 \lstinputlisting[style=py]{hit_target.py}
 Resultat etter kjøring av løsningsforslaget:
 \begin{verbatim}
>>> Number of points using v0 = 15: 0
>>> Number of points using v0 = 16: 1
>>> Number of points using v0 = 19: 1
>>> Number of points using v0 = 22: 2
 \end{verbatim}
 Noen delresultater som kan være til nytte:
 \begin{itemize}
 	\item Når $v_0 = 15$, så er $y = 2.9659$
 	\item Når $v_0 = 16$, så er $y = 3.03057617187$
 	\item Når $v_0 = 19$, så er $y = 3.16711218837$
 	\item Når $v_0 = 22$, så er $y = 3.25170971074$
 \end{itemize}


\newpage
\section*{Oppgave 3.5}
Poenget med denne oppgaven er å lage to funksjoner som i utgangspunktet skal returnere samme verdi, men fra ulike parametere. Testfunksjonen skal benytte ligningen i oppgaven til å konvertere en gitt tid til posisjon, slik at de to funksjonene kan sammenlignes.

I oppgave c) er det best å sjekke typen til \texttt{t} i en if/else block, og løse problemet helt separat.
\lstinputlisting[style=py]{cliff_throw.py}


\newpage
\section*{Oppgave 3.6}
Denne er nok i det tøffere laget fordi det er mye å holde styr på. Her er det meningen at de skal vise hva som er viktig å skille mellom ved for- og while loop, og ved hvilke tilfeller de skal brukes til. Tenker at de viser veldig god beherskelse på pensumet fram til nå om de får til b). 
\lstinputlisting[style=py]{block_frictions2.py}
Delresultat som kan være til nytte:\\
Listen \texttt{lengths} har verdiene\texttt{[2.0551774003489998, 4.2473666285, 2.8315777446474995, 6.371049942749998]}
\end{document}


