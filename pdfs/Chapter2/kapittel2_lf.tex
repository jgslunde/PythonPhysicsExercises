\documentclass[10pt,a4paper]{article}

\usepackage[utf8x]{inputenc}
\usepackage[norsk]{babel}
\usepackage[T1]{fontenc,url}
\usepackage[hang,small,bf]{caption}
\usepackage{relsize}
\usepackage{setspace}
\usepackage{parskip}
\usepackage{lmodern}
\usepackage{microtype}
\usepackage{verbatim}
\usepackage{amsmath, amssymb, amsthm}
\usepackage{mathtools}
\usepackage{tikz}
\usepackage{physics}
\usepackage{algorithm}
\usepackage{algpseudocode}
\usepackage{listings}
\usepackage{enumerate}
\usepackage{graphicx}
\usepackage{float}
\usepackage{hyperref}
\usepackage{varioref}
\usepackage{todonotes}
\usepackage{color}
\usepackage{siunitx}
\usepackage[margin=1.5cm]{geometry}
\labelformat{equation}{ligning~(#1)}

\renewcommand{\exp}{\mathrm{e}^}
\newcommand{\halflife}{t_{\frac{1}{2}}}

\definecolor{light_green}{rgb}{0, 0.6, 0}
\definecolor{light_grey}{rgb}{0.5, 0.5, 0.5}
\definecolor{magenta}{rgb}{0.7, 0, 0.5}


\lstdefinestyle{py}{
    language = python,
    frame = single,
    showstringspaces = false,
    basicstyle = \small\ttfamily,
    breaklines = true,
    commentstyle = \color{light_grey},
    keywordstyle = \color{magenta},
    stringstyle = \color{light_green},
}


\begin{document}


\section*{Oppgave 2.1}
Det viktige i denne oppgaven, er at de får kjennskap til hvordan en while loop skal implementeres. Det er viktig at det ikke testes for liket i loopen. 
\lstinputlisting[style=py]{throw_ball_height.py}
Resultat:
\begin{verbatim}
>>> It took the ball approximately 1.01 seconds to pass 5 meters
\end{verbatim}

\section*{Oppgave 2.2}
Hensikten med denne oppgaven er å bli kjent med hvordan man bruker for loops og print formatering til å lage tabeller. Formateringen bør altså helst være ryddig og i rette kolonner, og det skal brukes vitenskapelig notasjon. Det bør helst ikke brukes for mange eller for få desimaler i printen (~2-5). Det er helt greit å printe hastigheten enten i meter per sekund og i andeler av lyshastigheten, men bevegelsesmengde bør helst være i $\si{kg.m/s^2}$.

Pass på at $v^2/c^2$ i teorien kan gi float division.


\lstinputlisting[style=py]{relativistic_momentum.py}


\newpage
\begin{verbatim}
   Vel.(m/s)     Mom.(kgm/s)
          0               0
      3e+07         3.6e+10
      6e+07         7.2e+10
      9e+07        1.08e+11
    1.2e+08        1.44e+11
    1.5e+08         1.8e+11
    1.8e+08        2.16e+11
    2.1e+08        2.52e+11
    2.4e+08        2.88e+11
    2.7e+08        3.24e+11

     Vel.(c)    Clas.Mom.(kgm/s)     Rel.Mom.(kgm/s)
         0c                   0                   0
       0.1c             3.6e+10           3.618e+10
       0.2c             7.2e+10           7.348e+10
       0.3c            1.08e+11           1.132e+11
       0.4c            1.44e+11           1.571e+11
       0.5c             1.8e+11           2.078e+11
       0.6c            2.16e+11             2.7e+11
       0.7c            2.52e+11           3.529e+11
       0.8c            2.88e+11             4.8e+11
       0.9c            3.24e+11           7.433e+11
\end{verbatim}




\section*{Oppgave 2.3}

Dette er ideelle omstendigheter for å bruke en while loop, og bør ikke implementeres med for loop. Pass på at enten første eller siste element i listen må settes utenfor loopen, ettersom vi skal sette en verdi mer enn loopens lengde. 

Oppgave c) inkluderer litt komplisert indeksering, med mange mulige løsninger. Noen er selvsagt penere enn andre, men det viktigste er at det fungerer. I koden under ligger det to mulige løsninger. Den første baserer seg på å loope over indekser, og den andre å loope over selve elementene i listene.

\lstinputlisting[style=py]{radioactive_list.py}

\begin{verbatim}
>>> 48.87% is left of the carbon-11 after 1260 seconds.
\end{verbatim}

\begin{verbatim}
>>> Estimated half-life from simulation = 1260.00 s
>>> Actual half-life from analytical solution = 1219.94 s
\end{verbatim}


\begin{verbatim}
 t [Seconds]   N(t) [Kg]
          0         4.5
         60       4.349
        120       4.203
        180       4.063
        240       3.926
        300       3.795
        360       3.668
        420       3.545
        480       3.426
        540       3.311
        600         3.2
        660       3.093
        720       2.989
        780       2.889
        840       2.792
        900       2.699
        960       2.608
       1020       2.521
       1080       2.436
       1140       2.355
       1200       2.276
       1260       2.199
\end{verbatim}


\section*{Oppgave 2.4}
Viktig at summeringen blit utført og at de passer på å ikke gjøre helltalsdivisjon. 

Her skal det ikke være nødvendig å lagre massene og avstandene i lister. Det hadde vært flott om at det ble tydelig sagt i fra om at lister \textbf{ikke er nødvendige} dersom du ser noen gjøre dette. 
\lstinputlisting[style=py]{newton_gravitation.py} 
Den totale kraften $F$ skal bli: \\
\texttt{>>> The total attractive force on the object with mass M = 3 is 4.65492e-10 N}
\pagebreak


\section*{Oppgave 2.5}
Her er det bare å sette inn formelen gitt i oppgaven, og iterere over den fra $i=1$ til $i=29$. Det er fort gjort å gjøre feil på disse indeksene.

Det er også mulig å gange ut det faktoriserte $i$-leddet i formelen før man implementerer den, så keep that in mind hvis formelen i koden ser ukjent ut.

Det er også mulig å finne den totale energien ved å summe listen etter loopen.

\lstinputlisting[style=py]{quantum_trap.py}

Rettehjelp:\\
\texttt{energies[0] = 1.807e-15}\\
\texttt{energies[-1] = 3.554e-14}\\
\texttt{total\_energy = 5.415e-13}

\section*{Oppgave 2.6}
Her skal de vise at de forstår hvordan de kan oversette en while loop til en for loop. En elementvis for loop som den siste er ikke nødvendig, men gir litt penere kode. 
\lstinputlisting[style=py]{for_loop_over_loops.py}
Resultat:
\begin{verbatim}
>>> Least speed to complete the loop: 18.53 km/h
>>> Least speed to complete the loop: 20.88 km/h
>>> Least speed to complete the loop: 26.73 km/h
>>> Least speed to complete the loop: 30.04 km/h
\end{verbatim}













\end{document}
