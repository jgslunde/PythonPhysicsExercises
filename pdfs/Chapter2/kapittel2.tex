\documentclass[10pt,a4paper]{article}
 
\usepackage[utf8x]{inputenc}
\usepackage[norsk]{babel}
\usepackage[T1]{fontenc,url}
\usepackage[hang,small,bf]{caption}
\usepackage{relsize}
\usepackage{setspace}
\usepackage{parskip}
\usepackage{lmodern}
\usepackage{microtype}
\usepackage{verbatim}
\usepackage{amsmath, amssymb, amsthm}
\usepackage{mathtools}
\usepackage{tikz}
\usepackage{physics}
\usepackage{algorithm}
\usepackage{algpseudocode}
\usepackage{listings}
\usepackage{enumerate}
\usepackage{graphicx}
\usepackage{float}
\usepackage{hyperref}
\usepackage{varioref}
\usepackage{siunitx}
\usepackage{todonotes}
\usepackage{color}
\usepackage[margin=3cm]{geometry}
\labelformat{equation}{ligning~(#1)}
 
\renewcommand{\exp}{\mathrm{e}^}
\newcommand{\halflife}{t_{\frac{1}{2}}}
\newcommand{\half}{\frac{1}{2}}
\newcommand{\planck}{$h = \SI{6.626e-34}{J.s}$}
 
\definecolor{light_green}{rgb}{0, 0.6, 0}
\definecolor{light_grey}{rgb}{0.5, 0.5, 0.5}
\definecolor{magenta}{rgb}{0.7, 0, 0.5}
 
 
\lstdefinestyle{py}{
    language = python,
    frame = single,
    showstringspaces = false,
    basicstyle = \small\ttfamily,
    breaklines = true,
    commentstyle = \color{light_grey},
    keywordstyle = \color{magenta},
    stringstyle = \color{light_green},
}
 
 
\begin{document}
 \tableofcontents
 
	\section*{Oppgave 2.1 - Måle tid}
	 \addcontentsline{toc}{section}{Oppgave 2.1 - Måle tid - \texttt{throw\_ball\_height.py}}
	En ball slippes ned fra et stup i et rettlinjet bevegelse ved høyde $h$. Posisjonen til ballen etter en tid $t$ kan uttrykkes ved 
	\[
	y(t) = v_0t - \frac{1}{2}at^2 + h
	\]
	der $a$ er akselerasjonen i $\mathrm{m/s^2}$, $h$ er høyden (i meter) der ballen slippes fra og $v_0$ er startfarten (målt i m/s) .
	
	Vi ønsker å finne ut ved hvilken tid $t_1$ ballen passerer en spesifikk høyde $y_1$. Vi ønsker altså å finne verdien til $t_1$ slik at $y(t_1) = y_1$. Ballens posisjon blir målt per $\Delta t$ sekund. 
	
	Lag et program som finner ut hvor lang tid det tar ($t_1$) før ballen passerer høyden $y_1$. Programmet skal bruke en while loop. 
	
	Her lar vi $h = 10\, \si{\meter}$, $y_1 = 5\,\si{\meter}$, $\Delta t = 0.01\,\si{\second}$, $v_0 = 0\,\si{\meter.\per\second}$ og $a = 9.81\,\si{\meter.\per\square\second}$ .  
	
	\textbf{Hint:} Vi kan ikke bruke '\texttt{==}' i while loopen for å finne tiden $t_1$  når ballen er ved høyden $y_1$. Dette er fordi vi måler tiden pr. $\Delta t$ sekunder. Vi vil sannsynligvis måle posisjonen til ballen \textit{etter} den har nådd $y_1$. Det beste vi kan gjøre, er å øke tiden med $\Delta t$ så lenge høyden til ballen er \textit{større} enn $y_1$. Med en gang ballen har en høyde som er mindre eller lik $y_1$ ved en tid, så kan vi sette denne tiden til å være $t_1$. Programmet vårt vil derfor være noe unøyaktig, men dette er det beste vi kan få til siden vi ikke kan få uendelig liten $\Delta t$. 
	
	Filnavn: \texttt{throw\_ball\_height.py}
 
 
\section*{Oppgave 2.2 - Relativistisk bevegelsesmengde}
 \addcontentsline{toc}{section}{Oppgave 2.2 - Relativistisk bevegelsesmengde - \texttt{relativistic\_momentum.py}}
Fra klassisk fysikk har vi at bevegelsesmengden $p$ til et objekt med masse $m$ og hastighet $v$ er gitt ved
\begin{align*}
p = m\cdot v
\end{align*}
En satellitt med masse $\SI{1200}{kg}$ er fanget i nærheten av et sort hull, og akselererer raskt fra $v = 0$ til $v = 0.9 c$, der $c$ er lyshastigheten i vakuum, som er $c\approx \SI{3e8}{\m.\per \s}$.
 
\subsection*{a)}
Skriv et program som printer to pene kolonner til terminalen: Satellittens hastighet i intervaller på $0.1 c$, og satellittens bevegelsesmengde ved disse hastighetene.
 
\textbf{Hint:} Bruk vitenskapelig notasjon, '\%e', når du skriver verdiene, for å slippe ekstremt store floats. Eventuelt '\%g', som velger notasjon for deg. Prøv å ikke få med unødvendig mange desimaler i printen.
 
 
\subsection*{b)}
I oppgave 1.5 så vi at bevegelsesmengde i spesiell relativitetsteori er definert som
\begin{align*}
p_{rel} = m\cdot v\cdot \gamma, \ \ \ \ \gamma = \frac{1}{\sqrt{1-\frac{v^2}{c^2}}}
\end{align*}
Dette er den faktiske bevegelsesmengden til et objekt, men vi skal nå se at den klassiske formelen er en veldig god tilnærming ved 'lave' hastigheter.
 
Utvid programmet ditt fra oppgave a, slik at det printer tre kolonner: Hastigheten, den klassiske bevegelsesmengden, og den relativistiske bevegelsesmengden.
 
Filnavn: \texttt{relativistic\_momentum.py}
 
 
 
\section*{Oppgave 2.3 - Radioaktiv liste}
  \addcontentsline{toc}{section}{Oppgave 2.3 - Radioaktiv liste - \texttt{radioactive\_list.py}}
\subsection*{a)}
Ta utgangspunkt i formelen for radioaktiv nedbrytning fra oppgave 1.2:
\begin{align*}
N(t) = N_0\exp{-t/\tau}
\end{align*}

Lag en while loop som fyller to lister: En med tidspunkter $t$, og en med verdier av $N(t)$ ved disse tidspunktene. Start i $t=0$ og bruk tidssteg på $\SI{60}{s}$. Loopen skal avbrytes når den gjenværende massen av stoffet er under 50\% av den opprinnelige. Vi ser fortsatt på en masse $N_0 = \SI{4.5}{kg}$ av karbon-11, med tidskonstant $\tau = \SI{1760}{s}$.
 
Fra oppgave 1.3 husker vi at \textit{halveringstiden} til et materie er den tiden det tar for halvparten av materiet å nedbrytes. Fordi vi avbrøt loopen når halvparten av materiet var igjen, bør det siste elementet i tidslisten din være halveringstiden $\halflife$ til karbon-11.
 
Sjekk at dette stemmer ved å sammenligne den siste $N$-verdien i listen din med halveringstiden som definert i oppgave 1.3. Husk at verdien du målte kan avvike på opptil et minutt, fordi vi brukte tidssteg på et minutt mellom hver måling.
 
 
\subsection*{b)}
Kombiner listene til en nøstet liste, \texttt{Nt}, slik at hvert element i listen \texttt{Nt} er et par av tilhørende $t$- og $N(t)$-verdier. For eksempel skal det første elementet \texttt{Nt[0]} være en liste av \texttt{[0,\ 4.5]}. Bruk den nye nøstede listen til å skrive ut et pent table av alle de tilhørende $t$ og $N(t)$ verdier til terminalen.
 
Filnavn: \texttt{radioactive\_list.py}
 
 
 
 
 
	\section*{Oppgave 2.4 - Newtons universelle gravitasjonslov}
	  \addcontentsline{toc}{section}{Oppgave 2.4 - Newtons universelle gravitasjonslov - \texttt{newton\_gravitation.py}}
	Newtons gravitasjonslov beskriver hvordan gravitasjonen som en tiltrekningskraft virker mellom to legemer:
	\begin{equation*}
		F = G\frac{m_1m_2}{r^2}
	\end{equation*}
	der $m_1$ og $m_2$ er massene til de to legemene som tiltrekker hverandre og $r$ er avstanden mellom dem. 
	
	Konstanten $G$ er gravitasjonskonstanten som har følgende verdi:
	\[
	G = \SI{6.674e-11}{\cubic\meter\per\kilogram\per\squared\second}
	\]
	
	
	Vi ser på et legeme med masse $M = 3$ kg som blir påvirket av $N$ legemer med masse $m_1,m_2,\dots,m_N$. Det $i$-te legeme har masse $m_i = \frac{i}{6}+2\,\mathrm{kg}$, og har avstand $r_i = \sqrt{\qty(\frac{i}{4})^2 + 10}$ meter fra legemet med masse $M$.  
	
	Antall legemer er $N = 10$.
	
	
	Skriv et program som beregner den totale kraften $\sum_{i = 1}^N F_i = F_1 + F_2 + ... + F_N$ som virker på legemet med masse $M$. 
	
	Filnavn: \texttt{newton\_gravitation.py}
	
 
 
 
\section*{Oppgave 2.5 - Fanget kvantepartikkel}
\addcontentsline{toc}{section}{Oppgave 2.5 - Fanget kvantepartikkel - \texttt{quantum\_trap.py}}
Kvantemekanikk er et område i fysikken som omhandler virkeligheten på veldig små skalaer. En av resultatene i kvantemekanikk er at partikler noen ganger bare har lov til å ha spesifikke energinivåer, og kan aldri har energier mellom disse nivåene. Partiklene må da hoppe fra ett energinivå til et annet.
 
Kvantemekanikk sier at en partikkel som er fanget i en veldig liten 'boks'\footnote{Et uendelig dypt kvadratisk potensial} av lengde $L$, har det bare lov til å ha energiene:
\[	E_n = \frac{n^2h^2}{8mL^2}, \ \ \ \ n = 1,2,3\dots
\]
hvor $m$ er partikkelens masse, og $h$ er plancks konstant, \planck.
 
Vi skal se på et elektron med masse $\SI{9.11e-31}{kg}$ som fanget i en boks av lengde $\SI{e-11}{m}$. Elektronet starter på det laveste energinivået, $E_1$ (IKKE $E_0$!), og hopper så oppover, ett energinivå om gangen. Hvert hopp fra et energinivå $E_i$ til $E_{i+1}$ vil kreve en energi
\[  E_{i+1} - E_{i} = \frac{((i+1)^2-i^2)h^2}{8mL^2}
\]
 
Skriv en for loop som beregner energien er nødvendig for hver steg, og lagrer hver verdi i en liste. Summer også opp den totale energien som kreves gjennom alle hoppene.
 
Filnavn: \texttt{quantum\_trap.py}
 
 
 
\section*{Oppgave 2.6 - Loope over radiuser til looper}
\addcontentsline{toc}{section}{Oppgave 2.6 - Loope over radiuser til looper - \texttt{for\_loop\_over\_loops.py}}
For at en skal kunne kjøre gjennom en loop, er det nødvendig å ha en fart som er større enn en viss grense. 
 
Anta en stuntperson skal kjøre gjennom en loop. For at stuntpersonen skal være i kontakt med loopen gjennom hele stuntet, må farten $v$ v\textit{minst} være:

\[
v = \sqrt{gr}
\]
Her er $g = 9.81\,\mathrm{m/s^2}$ og $r$ er radiusen til loopen (i meter). 
 
 
Du har blitt gitt et program som bruker en while loop for å iterere gjennom en liste av radiuser $r$ til ulike looper. For hver radius $r$ programmet iterere gjennom blir den minste farten $v$ beregnet og deretter vist til skjerm:
\lstinputlisting[style=py]{while_loop_over_loops.py}
Programmet heter \texttt{while\_loop\_over\_loops.py} som du kan finne her: (lenke til program).
 
Skriv om programmet slik at den bruker en for loop isteden. Endre også programmet slik at den ikke printer $v$ for hver radius i loopen, men heller printer verdiene i en egen for loop etter alle $v$ har blitt regnet ut. 
 
Filnavn: \texttt{for\_loop\_over\_loops.py}
 
\end{document}
 
 

