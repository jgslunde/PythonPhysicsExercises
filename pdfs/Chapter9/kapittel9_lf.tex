\documentclass[10pt,a4paper]{article}

\usepackage[utf8x]{inputenc}
\usepackage[norsk]{babel}
\usepackage[T1]{fontenc,url}
\usepackage[hang,small,bf]{caption}
\usepackage{relsize}
\usepackage{setspace}
\usepackage{parskip}
\usepackage{lmodern}
\usepackage{microtype}
\usepackage{verbatim}
\usepackage{amsmath, amssymb, amsthm}
\usepackage{mathtools}
\usepackage{tikz}
\usepackage{physics}
\usepackage{algorithm}
\usepackage{algpseudocode}
\usepackage{listings}
\usepackage{enumerate}
\usepackage{graphicx}
\usepackage{float}
\usepackage{hyperref}
\usepackage{varioref}
\usepackage{todonotes}
\usepackage{color}
\usepackage{siunitx}
\usepackage[margin=1.5cm]{geometry}
\labelformat{equation}{ligning~(#1)}

\renewcommand{\exp}{\mathrm{e}^}
\newcommand{\halflife}{t_{\frac{1}{2}}}

\definecolor{light_green}{rgb}{0, 0.6, 0}
\definecolor{light_grey}{rgb}{0.5, 0.5, 0.5}
\definecolor{magenta}{rgb}{0.7, 0, 0.5}


\lstdefinestyle{py}{
    language = python,
    frame = single,
    showstringspaces = false,
    basicstyle = \small\ttfamily,
    breaklines = true,
    commentstyle = \color{light_grey},
    keywordstyle = \color{magenta},
    stringstyle = \color{light_green},
}


\begin{document}

\section*{Oppgave 9.1}
Ideelt skal \texttt{ConstantAcceleration} brukes så mye som mulig. Dette betyr at den lagrer alle mulig verdier, og kalles på både i utregningen av posisjon og hastighet.
\lstinputlisting[style=py]{Jerk.py}


\newpage
\section*{Oppgave 9.2}
Det er viktig at de kaller på superklassen sin funksjon for å regne ut massetettheten. 
\lstinputlisting[style=py]{Solid.py}


\newpage
\section*{Oppgave 9.3}
Poenget her er at de skal bruke \texttt{Cylinder} sin \texttt{inertia} for å beregne treghetsmomentet til sylinderskallet. 
\lstinputlisting[style=py]{Moment_of_inertia.py}
\begin{verbatim}
>>> Cylinder has moment of inertia I = 1.40625
>>> Cylindrical shell has moment of inertia I = 2.8125
\end{verbatim}
\end{document}


