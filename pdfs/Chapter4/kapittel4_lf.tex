\documentclass[10pt,a4paper]{article}

\usepackage[utf8x]{inputenc}
\usepackage[norsk]{babel}
\usepackage[T1]{fontenc,url}
\usepackage[hang,small,bf]{caption}
\usepackage{relsize}
\usepackage{setspace}
\usepackage{parskip}
\usepackage{lmodern}
\usepackage{microtype}
\usepackage{verbatim}
\usepackage{amsmath, amssymb, amsthm}
\usepackage{mathtools}
\usepackage{tikz}
\usepackage{physics}
\usepackage{algorithm}
\usepackage{algpseudocode}
\usepackage{listings}
\usepackage{enumerate}
\usepackage{graphicx}
\usepackage{float}
\usepackage{hyperref}
\usepackage{varioref}
\usepackage{todonotes}
\usepackage{color}
\usepackage{siunitx}
\usepackage[margin=1.5cm]{geometry}
\labelformat{equation}{ligning~(#1)}

\renewcommand{\exp}{\mathrm{e}^}
\newcommand{\halflife}{t_{\frac{1}{2}}}

\definecolor{light_green}{rgb}{0, 0.6, 0}
\definecolor{light_grey}{rgb}{0.5, 0.5, 0.5}
\definecolor{magenta}{rgb}{0.7, 0, 0.5}


\lstdefinestyle{py}{
    language = python,
    frame = single,
    showstringspaces = false,
    basicstyle = \small\ttfamily,
    breaklines = true,
    commentstyle = \color{light_grey},
    keywordstyle = \color{magenta},
    stringstyle = \color{light_green},
}



\begin{document}
\section*{Oppgave 4.1}
Viktige punkter er å bruke en try-except-blokk og la programmet kjøre en egendefinert testfunksjon.
\lstinputlisting[style = py]{uncertainty_Heisenberg.py}
Delresultater som kan være til nytte:
\begin{itemize}
	\item $\dfrac{h}{4\pi} = 5.27280326463\cdot 10^{-35}$
	\item $\Delta x_1\cdot\Delta p_1 = 5.272805\cdot 10^{-35}$
	\item $\Delta x_2 \cdot \Delta p_2 = 5.2 \cdot 10^{-35}$
\end{itemize}




\newpage
\section*{Oppgave 4.2}
I denne oppgaven skal det settes opp et input system, som først ser etter verdier i terminalen, og spør etter dem fra brukeren hvis verdiene ikke finnes eller er ugyldige. Disse verdiene brukes så til å regne ut et par enkle funksjoner.

Det er ikke nødvendig å implementere $x(t)$ og $v(t)$ som funksjoner, men det er veldig anbefalt.

I oppgave b) bør det være en \texttt{try} statement for å initialisere alle variablene, og to \texttt{except} statements for de to mulige feilene som kan oppstå ved denne initialiseringen. Begge spør brukeren om å sette inn verdiene manuelt.

\lstinputlisting[style = py]{particle_accelerator.py}

\begin{verbatim}
>>> Electron has a position of 1321.98 m and a velocity of -43.74 m/s at time 15.00 seconds.
>>> Particle has a position of 3300.00 m and a velocity of 220.00 m/s at time 15.00 seconds.
>>> Particle has a position of 3301.08 m and a velocity of 220.14 m/s at time 15.00 seconds.
\end{verbatim}





\newpage
\section*{Oppgave 4.3}
Hva som er tilstrekkelig nøyaktig er selvsagt åpent for diskusjon, og ingenting er direkte feil, men en grense ett sted mellom $0.1c$ og $0.5c$ er å foretrekke.

Vi har her valgt å sette $\gamma = 1$ ved det klassiske tilfellet, og bruke samme formel, men det også helt greit å regne de to formlene direkte i if/else-blokken.

De bør helst vende seg til å bruke \texttt{raw\_input}, og konvertere til float, istedenfor å bruke \texttt{input}.

I oppgave b) er det bare å bytte input med terminal argumenter.

I oppgave c) skal de wrappe initialiseringen av variabler i en \texttt{try/except} block. Denne bør inneholde hånteringer av både \texttt{IndexError} og \texttt{ValueError}. I tillegg skal det være to \texttt{if/else} tester for å sjekke massen og hastigheten, som skal raise sine egne errors. Alle errors skal ha med forklarende tekst.

\lstinputlisting[style = py]{momentum_input.py}


\newpage
\section*{Oppgave 4.4}
\subsection*{a)}
Her er det viktig at de skriver et program som kan lese inn \textit{vilkårlig} antall verdier for massene, helningsvinkelene og friksjonskoeffisientene. 
Dersom noen glemmer å lukke filene, kan det være flott å si ifra om dette er viktig for å unngå feil i programmet.

\subsection*{b)}
Det er fint om de greier å skrive til fil hva hver verdi står for. 
\lstinputlisting[style = py]{slide_books_friction.py}
\newpage
\section*{Oppgave 4.5}
Det som er nytt her, er \texttt{IOError}, men ellers er alt ganske likt som i boka. Det er også viktig å passe på å lukke fila. 
\lstinputlisting[style = py]{newton_gravitation_file.py}
Resultat:
\begin{verbatim}
>>> Total force on object with mass 0.7 kg is 5.85559 N
\end{verbatim}
\end{document}


