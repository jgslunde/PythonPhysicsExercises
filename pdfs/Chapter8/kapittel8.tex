\documentclass[10pt,a4paper]{article}
 
\usepackage[utf8x]{inputenc}
\usepackage[norsk]{babel}
\usepackage[T1]{fontenc,url}
\usepackage[hang,small,bf]{caption}
\usepackage{relsize}
\usepackage{setspace}
\usepackage{parskip}
\usepackage{lmodern}
\usepackage{microtype}
\usepackage{verbatim}
\usepackage{amsmath, amssymb, amsthm}
\usepackage{mathtools}
\usepackage{tikz}
\usepackage{physics}
\usepackage{algorithm}
\usepackage{algpseudocode}
\usepackage{listings}
\usepackage{enumerate}
\usepackage{graphicx}
\usepackage{float}
\usepackage{hyperref}
\usepackage{varioref}
\usepackage{siunitx}
\usepackage{todonotes}
\usepackage{color}
\usepackage[margin=3cm]{geometry}
\labelformat{equation}{ligning~(#1)}
 
\renewcommand{\exp}{\mathrm{e}^}
\newcommand{\halflife}{t_{\frac{1}{2}}}
\newcommand{\half}{\frac{1}{2}}
\newcommand{\planck}{$h = \SI{6.626e-34}{J.s}$}
 
\definecolor{light_green}{rgb}{0, 0.6, 0}
\definecolor{light_grey}{rgb}{0.5, 0.5, 0.5}
\definecolor{magenta}{rgb}{0.7, 0, 0.5}
 
 
\lstdefinestyle{py}{
    language = python,
    frame = single,
    showstringspaces = false,
    basicstyle = \small\ttfamily,
    breaklines = true,
    commentstyle = \color{light_grey},
    keywordstyle = \color{magenta},
    stringstyle = \color{light_green},
}
 
 
\begin{document}
\section*{Oppgave 8.1 - Energikonservering}
\addcontentsline{toc}{section}{Oppgave 8.1 - Energikonservering - \texttt{check\_energy\_conservation.py}}

Nå skal vi se på hvordan vi kan bruke generering av tilfeldige tall for å bekrefte et fysisk fenomen. Dette fenomenet som vi skal se på, er \textit{energikonservering}. Energikonservering brukes ofte når vi løser problemer i fysikk. Et legemes totale energi $E$ kan deles inn i to hovedkategorier: kinetisk energi $E_k$ og 
potensiell energi $E_p$. Det finnes også flere former for energi, men vi ser bort i fra disse i denne oppgaven. 
 
Vi har da at legemets totale energi $E$ kan skrives som $E = E_k + E_p$. 
Dersom vi har konservering av energi, så betyr det at for \textit{uansett} tid $t$ vi måler legemets energi ved vil energien $E$ være den samme. Vi kan skrive dette som
\begin{align*}
E_0 &= E_1 \\
E_{k,0} + E_{p,0} &= E_{k,1} + E_{p,1}
\end{align*}
der $E_0$ og $E_1$ er de totale energiene målt ved de ulike tidene $t_0$ og $t_1$. 
 
Kinetisk energi og potensiell energi for et legeme som blir kun påvirket av gravitasjonskraften er
\begin{align*}
E_k &= \frac{1}{2}mv(t)^2 \\
E_p &= mgy(t)
\end{align*}
 
Vi ser på en ball som blir kastet rett opp. Høyden $y(t)$ og farten $v(t)$ til ballen, $y(t)$, ved en tid $t$ kan vi finne ved
\begin{align*}
y(t) &= -gt^2 + v_0t \\
v(t) &= -gt + v_0
\end{align*}
der $g = 9.81\,\mathrm{m/s^2}$ og $v_0\,$m/s er kastefarten. 
 
Lag en testfunksjon som tar inn $N$, $m$ og $v_0$ som parametre og sjekker om vi har energikonservering. \\
La funksjonen generere $N$ tilfeldige verdier for tiden $t_0$ og $N$ tilfeldige verdier for tiden $t_1$. Verdiene skal ligge mellom 0 og $\frac{v_0}{g}$. \\
For alle verdier av $t_0$ skal programmet regne ut $E_0$, og tilsvarende skal programmet regne ut $E_1$ for alle verdier av $t_1$. \\
Sjekk så om verdiene er omtrent like, og bruk \texttt{assert} for å avbryte programmet og skrive ut passende feilmelding dersom vi ikke har energikonservering. 
 
Sett $N = 100$, $m = 0.057\,$kg og $v_0 = 17\,$m/s og kall på testfunksjonen for disse parametrene. 
 
\textbf{Hint:} Mye av koden kan vektoriseres - se det nederste eksempelet på s.510 i boken  for hvordan testing kan vektoriseres. 
 
Filnavn: \texttt{check\_energy\_conservation.py}



\section*{Oppgave 8.2 - Tilfeldig nedbrytning}
\addcontentsline{toc}{section}{Oppgave 8.2 - Tilfeldig nedbrytning - \texttt{random\_decay.py}}
Vi har tidligere studert hvordan en mengde radioaktivt materiale nedbrytes over tid, fra en rent analytisk formel, gitt som
\begin{align}
N(t) = N_0\exp{-t/\tau}
\label{eqn:8.2}
\end{align}
 
Med vår nye kjennskap til tilfeldige tall kan vi løse problemet på en mer numerisk måte, ved å modellere hvert atom individuelt.
 
Hvert atom i et radioaktivt materiale har en sjanse $p$ for å nedbrytes hvert sekund. Vi kan modellere materialet som en array, hvor hvert element representerer et atom. Vi gir hver atom en verdi $1$, som representerer at atomet ikke er nedbrutt, eller en verdi $0$.
 
Vi skal se på et isotop av nitrogen, \textit{nitrogen-16}, hvor hvert atom har en sjanse $p = 0.0926$ (eller $9.26\%$) for å nedbrytes hvert sekund.
 
\subsection*{a)}
Vi skal først studere det første sekundet av nedbrytningen. Lag et array av lengde 40 med verdier $1$ (som representerer 40 atomer som ikke er nedbrutt), og loop over hvert element i arrayet for å bestemme om hvert atom nedbrytes til $0$ eller ikke. Du kan gjøre dette ved å sammenligne tallet $p$ med et tilfeldig generert tall i intervallet $[0,\ 1]$.
 
Print det nye arrayet, og bekreft at noen av atomene faktisk ble nedbrutt til $0$.
 
\subsection*{b)}
Repeter prosessen over 20 sekunder ved å legge en til for loop rundt hele koden.
 
Implementer også et nytt array som lagrer hvor mange atomer som gjenstår etter hver iterasjon. Du kan gjøre dette ved med å bruke funksjonen \texttt{numpy.sum()} på arrayene\footnote{Denne funksjonen summerer opp alle verdiene i arrayet. Dette betyr at nedbrutte atomer teller som 0, og gjenværende som 1.}. Få også med tidspunkt $t=0$ i arrayet, som er før første iterasjon i loopen.
 
\subsection*{c)}
Plot mengden gjenværende atomer over tid. Inkluder også den analytiske løsningen for nedbrytingen av nitrogen-16, gitt av \ref{eqn:8.2}. Sett $N_0$ lik antallet atomer, og $\tau = \SI{10.3}{s}$ (som er et resultat av valget av $p$).
 
Prøv å øke antallet atomer i simuleringen, og (forhåpentligvis) observer at den numeriske løsningen blir mer og mer lik den analytiske.
 
Filnavn: \texttt{random\_decay.py}
 
 
 
 
 
 
\section*{Oppgave 8.3 - Optimale skytevinkler}
\addcontentsline{toc}{section}{Oppgave 8.3 - Optimale skytevinkler - \texttt{optimal\_angles\_shoot.py}}
Du har laget en liten ballkanon som skal treffe et målt område på en vegg (har du løst 3.3 i dette heftet tidligere, kan du godt gjenbruke og modifisere koden i denne oppgaven). Området begynner ved en høyde $h_0$ og slutter ved en høyde $h_1$ på veggen. 
Høyden til ballen kan modelleres ved
\begin{align*}
y(t) =  -\frac{1}{2}g t^2 + v_0t\sin\theta
\end{align*}
der $g = 9.81\;\mathrm{m/s^2}$, $v_0\:$m/s skytefarten og $\theta$  skytevinkelen. 
Ballen treffer veggen ved en tid $T = \dfrac{b }{v_0\cos\theta}$, der $b$ er avstanden (i meter) mellom kanonen og veggen. 
 
\subsection*{a)}
Lag et program som genererer $N$ verdier av $\theta$ ved bruk av Pythons \texttt{uniform(0,$\pi$)} eller Numpys \texttt{uniform(0,$\pi$,$N$)}.\\
Verdiene for $\theta$ skal sendes inn som parameter til en funksjon som sjekker om kanonen treffer innenfor området ved de gitte $\theta$-ene. De $\theta$-ene som gir at ballen treffer treffer det malte området skal bli lagret i en liste eller array. Listen eller array-et skal så returneres av funksjonen. 
 
Sett $N = 1000$, $h_0 = 3\,$m, $h_1 = 3.25\,$m, $b = 3.5\,$m og $v_0 = 25\,$m/s og kjør programmet for disse parametrene. 
\subsection*{b)}
Bruk listen eller array-et som ble returnert fra funksjonen i a) for å finne gjennomsnittet $\theta'$ av de vinklene som fikk ballen til å treffe det avgrensede området. Hvis listen er tom, skal programmet generere nye $N$ verdier for $\theta$ helt til den returnerte listen eller array-et ikke er tom. Dette kan for eksempel gjøres ved å ha en while loop som vil kjøre så lenge listen er tom. 
 
La programmet gjøre det samme som i a) med den forskjell i at verdiene for $\theta$ blir generert ved \texttt{random.gauss($\theta'$,0.05)} eller \texttt{np.random.uniform($\theta'$,0.05,$N$)}. 
 
La programmet ditt skrive ut hvor mange vinkler fra a) og fra denne deloppgaven som traff området. Hvordan er antallet vinkler som treffer området sammenlignet med antallet som ble generert i a)?
 
\textbf{Kommentarer til resultatet i b):} Verdiene av $\theta$ i a) ble generert slik at verdiene mellom 0 og $\pi$ var like sannsynlige. I b) derimot, ble verdiene generert ved å bruke en \textit{normal fordeling} sentrert om $\theta'$. 

Normalfordelingen har en bjelleform og forteller oss hvor sannsynlig det er for at vi trekker verdier i et område om $\theta'$.  
Verdien til $\theta'$ vil være mest sannsynlig å få, mens sannsynligheten til å få andre verdier vil minke jo større differansen er mellom verdiene og $\theta$. Hvor brått sannsynligheten vil synke, er også avhengig av en ekstra verdi, kalt standardavviket, som i dette tilfellet er 0.05 (som er funnet ved litt eksperimentering).  

Jo større standardavvik en bruker, jo mest sannsynlig blir de andre verdiene, og mindre standardavvik vil gi at de andre verdiene blir mindre sannsynlige å få. 
 
Filnavn:\texttt{optimal\_angles\_shoot.py}
 


\section*{Oppgave 8.4 - Varm gass}
\addcontentsline{toc}{section}{Oppgave 8.4 - Varm gass - \texttt{gaussian\_velocities.py}}
I denne oppgaven skal vi studere hastigheten til partikler i en varm gass. Temperatur er egentlig bare tilfeldig uorganisert bevegelse. Hvis gassen står stille, vil selvsagt den totale bevegelsen til alle partiklene i gassen være 0, men det forekommer fortsatt mye intern bevegelse.
 
Partikler i en gass med absolutt temperatur $T$ har hastigheter som er normalfordelt (Gaussisk fordelt) rundt et gjennomsnitt på 0, med et standardavvik \footnote{Standardavviket representerer bredden på fordelingskurven, og et høyere standardavvik bestyr større hastighet, og høyere temperatur.} $s = \sqrt{\frac{kT}{m}}$, hvor $k = \SI{1.38e-23}{m^2.kg.s^{-2}.K^{-1}}$ er Boltzmann's konstant. Denne fordelingen gjelder individuelt for hastigheten langs hver akse, $(v_x,v_y,v_z)$.
 
Vi kan generere tilfeldig normalfordelte tall med numpyfunksjonen \texttt{numpy.normal(m,s,shape)}, hvor \texttt{shape} er dimensjonene til arrayet at tilfeldige tall. For å representere \texttt{N} partikler med normalfordelt hastighet langs hver akse, trenger vi et array på formen \texttt{shape = (N,3)}. Dette betyr at hvert element i arrayet representerer en enkelt partikkel, som i seg selv er en array av 3 elementer, som igjen representerer hastigheten langs hver akse.
 
\subsection*{a)}
Set $N = 20$, $m = \SI{e-22}{kg}$, $T = \SI{300}{K}$, bruk numpyfunksjonen til å lagre et array av normalfordelte hastigheter.\\
Bruk \texttt{random.choice()} funksjonen til å velge ut en tilfeldig av de 20 partiklene, og print hastighetene til partikkelen til terminalen.
 
\subsection*{b)}
Den totale kinetiske energien til $N$ partikler i en gass av temperatur $T$ er gitt som
\[	E_k = \frac{3}{2}kTN
\]
 
En alternativ metode for å regne ut den kinetiske energien til gassen er å summere opp den kinetiske energien til hver enkelt partikkel, som er gitt som
\[	E_k = \frac{1}{2}mv^2
\]
Loop over hver partikkel, og beregn deres kinetiske energi ved å sette absoluttverdien av hastigheten, $|v| = \sqrt{v_x^2 + v_y^2 + v_z^2}$, inn i formelen over. Sammenlign resultatet med det du fikk fra den øverste formelen.
 
Filnavn: \texttt{gaussian\_velocities.py}
 
 
 
\section*{Oppgave 8.5 - Regndråper}
\addcontentsline{toc}{section}{Oppgave 8.5 - Regndråper - \texttt{raindrops.py}}
Vi skal se på \textit{terminalfarten} til mange regndråper. Terminalfarten er farten en regndråpe har når tyngdekraften trekker dråpen ned like mye som den møter luftmotstand. 
 
For en regndråpe med radius $r\,$meter der vi antar dråpen har en perfekt kuleform, kan vi regne ut at terminalfarten $v_T$ (i \si{\meter.\per\second}) vil være
\[
v_T = \sqrt{\frac{8r\rho_v g}{3\rho C}}
\]
der $r\,\si{\meter}$ er radiusen til dråpen, $\rho_v = 1\,\si{g.\per\cubic\centi\meter} = 1000 \,\si{kg.\per\cubic\meter}$ er den omtrentlige tettheten til vann, $g = 9.81 \,\si{\meter.\per\square\second}$, $\rho = 1.293 \,\si{\kg.\per\cubic\meter}$ er omtrentlige tettheten til luft og $C = 0.47$ betegner hvor mye motstand regndråpen utgjør på sin omgivelse (i dette tilfellet: luften).
 
\subsection*{a)}
I denne deloppgaven skal vi \textit{ikke} bruke Numpy-modulen.
 
Lag et program som generer $N = 100000$ ulike regndråper representert ved tilfeldige verdier for radius i intervallet $[1\,\si{\milli\meter}, 6\,\si{\milli\meter}) = [10^{-3}\,\si{\meter},6\cdot 10^{-3}\,\si{\meter}) $.
 
 Programmet skal finne ut hvor lang tid det tar for å generere de $N$ regndråpene og beregne gjennomsnittet av terminalfartene til dem. 
 
 Programmet skal tilslutt skrive ut den gjennomsnittlige terminalfarten og hvor lang tid programmet brukte. 
 
 \subsection*{b)}
Utvid programmet ditt slik at samme beregninger som i a) utføres, men nå med bruk av Numpy-modulen og vektorisering. 
Skriv ut den gjennomsnittlige terminalfarten sammen med tidsbruket programmet brukte ved vektorisering og bruk av Numpy-modulen. 
 
I denne deloppgaven vil du også kunne se en betydelig forskjell i tidsbruk sammenlignet med tidsbruket fra a). 
 
\textbf{Hint:} For å ta tiden i sekunder til et program, kan du bruke \texttt{time}-modulen. Et eksempel på en kode som finner ut hvor lang tid programmet bruker for å skrive 'Hello, world!' til skjerm, er:
\lstinputlisting[style=py]{tatid_pseudokode.py}
 
Filnavn: \texttt{raindrops.py}
\end{document}
 
 

